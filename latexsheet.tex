\documentclass[10pt,landscape]{article}
\usepackage{multicol}
\usepackage{calc}
\usepackage{amssymb}
\usepackage{ifthen}
\usepackage{array}
\usepackage[landscape]{geometry}
\usepackage{hyperref}
\usepackage[fontsize=6pt]{fontsize}
\usepackage{amsmath}
% To make this come out properly in landscape mode, do one of the following
% 1.
%  pdflatex latexsheet.tex
%
% 2.
%  latex latexsheet.tex
%  dvips -P pdf  -t landscape latexsheet.dvi
%  ps2pdf latexsheet.ps


% If you're reading this, be prepared for confusion.  Making this was
% a learning experience for me, and it shows.  Much of the placement
% was hacked in; if you make it better, let me know...


% 2008-04
% Changed page margin code to use the geometry package. Also added code for
% conditional page margins, depending on paper size. Thanks to Uwe Ziegenhagen
% for the suggestions.

% 2006-08
% Made changes based on suggestions from Gene Cooperman. <gene at ccs.neu.edu>


% To Do:
% \listoffigures \listoftables
% \setcounter{secnumdepth}{0}


% This sets page margins to .5 inch if using letter paper, and to 1cm
% if using A4 paper. (This probably isn't strictly necessary.)
% If using another size paper, use default 1cm margins.
\ifthenelse{\lengthtest { \paperwidth = 11in}}
	{ \geometry{top=.5in,left=.5in,right=.5in,bottom=.5in} }
	{\ifthenelse{ \lengthtest{ \paperwidth = 297mm}}
		{\geometry{top=1cm,left=1cm,right=1cm,bottom=1cm} }
		{\geometry{top=1cm,left=1cm,right=1cm,bottom=1cm} }
	}

% Turn off header and footer
\pagestyle{empty}
 

% Redefine section commands to use less space
\makeatletter
\renewcommand{\section}{\@startsection{section}{1}{0mm}%
                                {-1ex plus -.5ex minus -.2ex}%
                                {0.5ex plus .2ex}%x
                                {\normalfont\large\bfseries}}
\renewcommand{\subsection}{\@startsection{subsection}{2}{0mm}%
                                {.2ex}%
                                {.2ex}%
                                {\normalfont\normalsize\bfseries}}
\renewcommand{\subsubsection}{\@startsection{subsubsection}{3}{0mm}%
                                {-1ex plus -.5ex minus -.2ex}%
                                {1ex plus .2ex}%
                                {\normalfont\small\bfseries}}
\makeatother

% Define BibTeX command
\def\BibTeX{{\rm B\kern-.05em{\sc i\kern-.025em b}\kern-.08em
    T\kern-.1667em\lower.7ex\hbox{E}\kern-.125emX}}

% Don't print section numbers
\setcounter{secnumdepth}{0}


\setlength{\parindent}{0pt}
\setlength{\parskip}{0pt plus 0.5ex}


% -----------------------------------------------------------------------

\begin{document}

\raggedright
\footnotesize
\begin{multicols}{3}


% multicol parameters
% These lengths are set only within the two main columns
%\setlength{\columnseprule}{0.25pt}
\setlength{\premulticols}{1pt}
\setlength{\postmulticols}{1pt}
\setlength{\multicolsep}{1pt}
\setlength{\columnsep}{2pt}

\section{Data/R}
\subsection{Data Visualization}
\begin{tabular}{@{}ll@{}}
\verb!geom_hist and geom_density!    & distribution of numerical columns \\
\verb!geom_bar!  & number of occurences in a categorical col \\
\verb!geom_boxplot! & shape \& distribution of numerical vars \\
\verb!geom_scatter + geom_line*!  & numerical vs. numerical \\
\verb!geom_bar!  & bar plot for count of categorical vars \\
\verb!geom_hline(yintercept)!  & horizontal line\\
\verb!geom_vline(xintercept)!  & vertical line\\
\verb!geom_abline(slope, intercept)!  & linear function, requires  \\
\verb!geom_segment!  & straight line between (x, y) and (xend, yend) \\
\verb!geom_smooth!  & plots a line/curve of best fit \\
\end{tabular}

*\verb!geom_line! only makes sense with an ordering (e.g. the x-axis is year and observations connect together)

\subsection{Data Manipulation}
\newlength{\MyLen}
\settowidth{\MyLen}{\texttt{letterpaper}/\texttt{a4paper} \ }
\begin{tabular}{@{}p{\the\MyLen}%
                @{}p{\linewidth-\the\MyLen}@{}}
\verb!arrange(asc(col))!  & arranges $col$ by ascending order\\
\verb!arrange(desc(col))!  & arranges $col$ by descending order\\
\verb!relocate(data, col, .before, .after)!  & \hskip6.5em relocates a column relative to its neighbors*\\
\verb!arrange(desc(col))!  & arranges $col$ by descending order\\
\verb!slice(data, pos)!  & indexes rows\\
\verb!bind_rows(df1, df2, ...)!  & \hskip1.5em dfs w/ same columns, concats rows\\
\verb!bind_cols(df1, df2, ...)!  & \hskip1.5em dfs w/ same \# rows, concats cols, renames repeated cols\\
\verb!semi_join(x, y, by)!  & returns rows from x w/ matching val for \verb!by! in y\\
\verb!anti_join(x, y, by)!  & returns rows from x w/o a match in y\\
\verb!full_join(x, y, by)!  & standard outer join\\
\verb!left_join(x, y, by)!  & standard left join, \verb!x! is the left df\\
\verb!right_join(x, y, by)!  & standard right join, \verb!y! is the right df\\
\end{tabular}

*specifying no neighbors moves $col$ to leftmost col, specifyfing both is error \\
Suppose we have the following table \verb!fish_encounters!
\begin{tabular}{c|c|c}
        fish & station & seen \\
        \hline
        4842 & Release & 1 \\
        4842 & I80\_1 & 1 \\
        4842 & Lisbon & 1 \\
        4842 & Rstr & 1 \\
        4842 & Base\_TD & 1 \\
        4842 &  BCE  & 1 \\
        4842 & BCW & 1 \\
        4842 & BCE2 & 1 \\
        4842 & BCW2 & 1 \\
        4842 & MAE & 1 \\  
        4845 & BCE & 0 \\  
\end{tabular} \\
pivot\_wider(fish\_encounters, names\_from = station, values\_from = seen, values\_fill = 0)
\begin{tabular}{@{}c@{}|*{10}{@{}c@{}|}}
        Fish & Release & I80\_1 & Lisbon & Rstr & Base\_TD & BCE & BCW & BCE2 & BCW2 & MAE \\
        \hline
        1 & 4842 & 1 & 1 & 1 & 1 & 1 & 1 & 1 & 1 & 1 \\
        2 & 4843 & 1 & 1 & 1 & 1 & 1 & 1 & 1 & 1 & 1 \\
        3 & 4844 & 1 & 1 & 1 & 1 & 1 & 1 & 1 & 1 & 1 \\
        4 & 4845 & 1 & 1 & 1 & 1 & 0 & 0 & 0 & 0 & 0 \\
\end{tabular} \\
Suppose we have the following table \verb!billboard!
\begin{tabular}{*{10}{@{}c@{}|}}
        artist &    track & date.entered &  wk1  & wk2 &  wk3  & wk4 &  wk5 &  wk6 &  wk7 \\
        \hline
 2 Pac     & Baby… & 2000-02-26   &   87  &  82   & 72  &  77 &   87   & 94   & 99 \\
 2Ge+her  &  The … & 2000-09-02 &      91 &   87 &   92 &   NA  &  NA &   NA &   NA \\
 3 Doors D… & Kryp… & 2000-04-08  &  81 &   70 &   68 &   67 &   66   & 57  &  54 \\
 3 Doors D… & Loser & 2000-10-21 &      76 &    76 &   72  &  69  &  67 &   65 &   55 \\
 504 Boyz &  Wobb… & 2000-04-15 &     57  &  34 &   25 &   17 &   17 &   31 &   36 \\
\end{tabular} \\
pivot\_longer(billboard, cols = starts\_with("wk"), names\_to = "week", names\_prefix = "wk", values\_to = "rank", values\_drop\_na = TRUE)
\begin{tabular}{*{5}{@{}c@{}|}}
artist & track & date.entered & week & rank \\
\hline
2 Pac & Baby Don't Cry (Keep... & 2000-02-26 & 1 & 87 \\
2 Pac & Baby Don't Cry (Keep... & 2000-02-26 & 2 & 82 \\
2 Pac & Baby Don't Cry (Keep... & 2000-02-26 & 3 & 72 \\
2 Pac & Baby Don't Cry (Keep... & 2000-02-26 & 4 & 77 \\
2 Pac & Baby Don't Cry (Keep... & 2000-02-26 & 5 & 87 \\
2 Pac & Baby Don't Cry (Keep... & 2000-02-26 & 6 & 94 \\
2 Pac & Baby Don't Cry (Keep... & 2000-02-26 & 7 & 99 \\
2Ge+her & The Hardest Part Of ... & 2000-09-02 & 1 & 91 \\
2Ge+her & The Hardest Part Of ... & 2000-09-02 & 2 & 87 \\
2Ge+her & The Hardest Part Of ... & 2000-09-02 & 3 & 92 \\
\end{tabular}

\subsection{Dates \& Strings}
\settowidth{\MyLen}{\texttt{multicol} }
\begin{tabular}{@{}p{\the\MyLen}%
                @{}p{\linewidth-\the\MyLen}@{}}
\verb!ymd(), dmy(), ...!  & \hskip4.5em converts string to datetime according to order of y-m-d\\
\verb!wdate(date)!  & \hskip4.5em gets the day of the week for a given date\\
\verb!strc(str1, str2, ...)!  & \hskip5.5em concatenates strings/vectors of strings\\
\verb!str_detect(str, pattern)!  & \hskip6.5em TRUE if $\exists$ a substring of \verb!str! that matches \verb!pattern!\\
\verb!str_extract(str, pat, group)!  & \hskip8em finds 1st match in \verb!str! for \verb!pat!, \verb!group! takes matched pattern, returns text matching \verb!group!\\
\verb!str_extract_all(string, pattern)!  & \hskip9.5em returns all matches to \verb!pattern! \\
\verb!str_sub(string, start, end)!  & \hskip7.5em indexes into \verb!string! \\
\verb!str_count(string, pattern)!  & \hskip7em count \# of matches to \verb!pattern! in \verb!string!  \\
\end{tabular}

str\_replace(string, pattern, replacement), str\_replace\_all(string, pattern, replacement) - these exist \\
putting color, fill, alpha, etc. outside of aes(), i.e. typically inside of geom\_x() functions will set it as a constant for the whole graph \\
putting color, fill, alpha, etc. inside of aes() typically implies you have a column in your df (like year) that sets the groups appropriately \\
every geom\_x() function inherits the aes() from ggplot, unless they have their own aes() which overrides the ggplot \\
R always prints dates as YYYY-MM-DD

\subsection{Regex}
\settowidth{\MyLen}{\texttt{.author.text.} }
\begin{tabular}{@{}p{\the\MyLen}%
                @{}p{\linewidth-\the\MyLen}@{}}
\verb!\d! & digits \\
\verb!\s!  & whitespace \\
\verb!\w!   & alphabetic and numeral \\
\verb!^! & matches the start of each line \\
\verb!$! & matches the end of each line \\
\verb!?! & 0 or 1 \\
\verb!+! & 1 or more \\
\verb!*! & 0 or more \\
\verb!{n}! & exactly n \\
\verb!{n, }! & n or more \\
\verb!{n, m}! & between n and m \\

\end{tabular}
Capitalizing any of the above is the complement 

You can also create your own character classes using []:
\begin{tabular}{@{}p{\the\MyLen}%
        @{}p{\linewidth-\the\MyLen}@{}}
\verb![abc]! & matches a, b, or c \\
\verb![a-z]!  & matches every character between a and z \\
\verb![^abc]!   & matches anything except a, b, or c \\
\verb![\^\-]!   & matches \textasciicircum or - \\
\end{tabular}
Parenthesis make groups which can be backreferenced \\
pattern <- "(..)\textbackslash\textbackslash1" \#(..) is some pair of anything, and \\1 takes that same pair  \\
fruit \%>\% str\_subset(pattern) \\
"banana"      "coconut"     "cucumber"    "jujube"    "papaya" "salal berry"
\section{Basic Probability}
\subsection{Probability Theory}
For some random variable $X$, $E(X) = \sum_{x = 0}^{n} x * P(X = x)$. \\
The expected value is just the sum of each outcome multiplied by its porbability. \\
$Var(X) = E((X - \mu) ^ 2)$, $\mu = E(X)$ \\
Again, this is just multiplying the squared difference of the mean from each observation with each observation's respective probability, $sum((x-mu)^2*p)$.

Suppose that the distribution of $X$ is proportional with the function $g(x) = 6 - |x - 5|$. \\
Say that we have outcomes $1, 2, \cdots, 10$, this means $P(X = x) = a(6 - |x - 5|)$. \\
We know that the total number of outcomes and number of current outcomes must be proportional to the function. \\
The way to make the number of outcomes proportional is to find $\sum_{i = 1}^{10}6 - |i - 5|$. \\
To keep the possible values proportional, each probability is $\frac{j}{\sum_{i = 1}^{10}6 - |i - 5|}$, $j \in {g(1), g(2), \cdots, g(10)}$.

\subsection{Binomial Distributions}
Properties of Binomials
\begin{tabular}{@{}p{\the\MyLen}%
        @{}p{\linewidth-\the\MyLen}@{}}
\verb!b! & binary outcomes \\
\verb!i! & independence \\
\verb!n! & fixed sample size \\
\verb!s! & same probability \\
\end{tabular} \\
Binomial Formulas \\
\begin{tabular}{@{}p{\the\MyLen}%
        @{}p{\linewidth-\the\MyLen}@{}}
$\mu = np$ & $\sigma^2 = np(1 - p)$ \\
binom prob & $P(X = k) = \binom{n}{k} {p^k(1 - p)^{n-k}}$ \\
\end{tabular}
R Binomial Functions \\
\begin{tabular}{@{}p{\the\MyLen}%
        @{}p{\linewidth-\the\MyLen}@{}}
\verb!rbinom(n, size, prob)! & \hskip3em random binomial samples \\
\verb!dbinom(x, size, prob)! & \hskip3em density fcn at \verb!x! \\
\verb!qbinom(p, size, prob)! & \hskip3em get the smallest value in the \verb!q!th quantile \\
\verb!pbinom(q, size, prob)! & \hskip3em $P(X<= \verb!q!)$ \\
\verb!pbinom(q, size, prob, lower.tail = T)! & \hskip9.5em $1 - P(X<= \verb!q!) = p(X > \verb!q!)$ \\
\end{tabular}
Note that $\binom{n}{k} = \frac{n!}{k!(n-k)!}$
\subsection{Normal Distributions}
\settowidth{\MyLen}{\texttt{.begin.quotation.}}
R Normal Functions*
\begin{tabular}{@{}p{\the\MyLen}%
                @{}p{\linewidth-\the\MyLen}@{}}
\verb!pnorm(q, size, prob)! & \hskip3em $P(X< \verb!q!)$ \\
\end{tabular}
At any given $x$, $X \sim N(\mu, \sigma)$, $P(X = x) = 0$. \\
The standard normal is $X \sim N(0, 1)$. \\
Any normal has its z-scores as equivalent observations in the standard normal.
In other words, $X \sim N(\mu, \sigma) \implies Z = \frac{X - \mu}{\sigma} \sim N(0, 1)$. \\
*includes \verb!rnorm()!, \verb!qnorm()! which have same functionality as the binom fcns \\
Suppose that $\exists X \sim Binom(n, p)$, with $np(1 - p) \geq 10$ \\
Note the conditions this tests, $p$ can't be too close to 0 or 1 (causes skew), and $n$ must be sufficiently large (reduces variance). \\
We can approximate that binomial with $X \sim N(np, \sqrt{np(1 - p)})$. \\
Recall that this appoximation isn't perfect, the normal has an effect of "cutting off" the binomial distribution. \\
Correct for this with $P(X \leq x + .5)$ wheing finding $P(X \leq x)$, $P(X \geq x - .5)$ when finding $P(X \geq x)$ \\
As a general rule, \\
65\% of data 1 SD from the mean \\
95\% of data 2 SD from the mean \\
99\% of data 3 SD from the mean \\
\section{Inference}

\subsection{Inference on Proportions}
\settowidth{\MyLen}{\texttt{.begin.description.}}
Formulas ($\hat{P}$* is a random estimator for point estimate $\hat{p}$):
\begin{tabular}{@{}p{\the\MyLen}%
                @{}p{\linewidth-\the\MyLen}@{}}
$Var(\hat{P})$        &  $Var(\frac{X}{n}) = \frac{Var(X)}{n}$ \\
$SE(\hat{p})$          &  $\sqrt{Var(\hat{P})}$ \\
CI      & \hskip3em  $\hat{p} \pm z * SE$ \\
$z$      & \hskip3em  $qnorm(1 - \frac{p}{2})$* \\
\end{tabular}
*$\hat{P} \sim N(p, \sqrt{\frac{p(1-p)}{n}})$, this is still random\\
*where $p$ is the desired conf interval \\
Agresti-Coull Method (use $\tilde{p}$ in place of $\hat{p}$)\\
\begin{tabular}{@{}p{\the\MyLen}%
        @{}p{\linewidth-\the\MyLen}@{}}
$\tilde{p}$        &  $\frac{x + 2}{n + 4}$ \\
$SE(\tilde{p})$          &  $\sqrt{\frac{\tilde{p}(1 - \tilde{p})}{n + 4}}$ \\
CI      & \hskip3em  $\tilde{p} \pm z * SE$ \\
$z$      & \hskip3em  $qnorm(1 - \frac{p}{2})$* \\
\end{tabular} \\
*where $p$ is the desired conf interval \\
In theory this is a better estimate, still when $SE$ is too small the $CI$ can be too narrow. \\
Using $\tilde{p}$ moves the estimate closer to $.5$. \\
When $\hat{p}$ is closer to 0 or 1 than $p$, $SE$ tends to be underestimated, and vice versa for $\hat{p}$ closer to $.5$ than $p$. \\
Hypothesis testing - determine if a result we found was due to random chance
\begin{enumerate}
        \item Have a binomial model
        \item State $H_0$ and $H_A$
        \item Choose test statistic 
        \item Find p-value and see if it's under some $\alpha$
\end{enumerate}
Assume $H_0$ is true. Now find probability we observed a certain outcome. \\
Suppose $H_0 | X \sim Binom(n, k)$ and $H_0: k = .5$, $H_A: k \ne .5$. 
We observe $j$ successes and $n$ observations in total. Then $p$ is $2 * qnorm(j, n, .5)$, 
since the probability distribution is symmetric and $\ne$ necessitates a 2-sided test. 
$H_0$ also assumes a binomial distribution w/ chance of success being $.5$. \\
Differnce in Proportions\\
\begin{tabular}{@{}p{\the\MyLen}%
        @{}p{\linewidth-\the\MyLen}@{}}

$\bar{p}$          &  $\frac{x_1 + x_1}{n_1 + n_2}$ \\
$SE(\hat{p_1} -\hat{p_2})$          &  $\sqrt{\frac{\bar{p}(1-\bar{p})}{n_1} + \frac{\bar{p}(1-\bar{p})}{n_2}}$ \\
CI      & \hskip3em  $\tilde{p} \pm z * SE$ \\
$z$      & \hskip3em  $qnorm(1 - \frac{p}{2})$* \\
\end{tabular} \\
*where $p$ is the desired conf interval \\
Agresti-Coffe Method (use $\tilde{p}$ in place of $\hat{p}$)\\
\begin{tabular}{@{}p{\the\MyLen}%
        @{}p{\linewidth-\the\MyLen}@{}}
$\tilde{p}$        &  $\frac{x + 1}{n + 2}$ \\
$SE(\tilde{p_1} - \tilde{p_2})$          &  $\sqrt{Var(\tilde{p_1}) + Var(\tilde{p_2})}$ \\
$Var(\tilde{p_i})$          &  $\frac{\tilde{p_i}*(1 - \tilde{p_i})}{(n_i + 2))}$ \\
CI      & \hskip3em  $\tilde{p} \pm z * SE$ \\
$z$      & \hskip3em  $qnorm(1 - \frac{p}{2})$* \\
\end{tabular} \\
Hypothesis Testing Difference in Proportions - determine if result was due to random chance
\begin{enumerate}
        \item Have 2 binomial models
        \item State $H_0$ and $H_A$
        \item Choose test statistic 
        \item Find p-value and see if it's under some $\alpha$
\end{enumerate}
Say that $H_0: p_1 - p_2 = 0$, $H_A: p_1 \ne p_2$. Say that the number of success is $i_1$ and $i_2$ respectively. \\
Since we test $p_1 - p_2$ the differnces will be normally distributed. \\
Estimate the combined probability as $\bar{p} = \frac{i_1 + i_2}{n_1 + n_2}$. \\
Again, assuming $H_0$ is true calculate $z$. $z = \frac{(\hat{p_1} - \hat{p_2}) - (p_1 - p_2)}{SE}$. THIS $p_1 - p_2$ IS THE $p_1 - p_2$ DEFINED BY $H_0$. \\
 $p$ is the area area under the standard normal, or $2 * P(X > z)$ in this case.

\subsection{Inference on Means}
\settowidth{\MyLen}{\texttt{.pageref.marker..}}
Formulas:
\begin{tabular}{@{}p{\the\MyLen}%
                @{}p{\linewidth-\the\MyLen}@{}}
\verb!CI!   &   $z * SE$  \\
$z$   &   $qt(p, n - 1)$  \\
$SE(\bar{x})$   &   $\frac{\sigma}{\sqrt{n}}$  \\
$T$ & $\frac{\bar{X} - \mu_0}{s / \sqrt{n}}$* \\
$s$ & $\sqrt{\frac{\sum_{i = 1}^{n}(x_i - \bar{x})^2}{n - 1}}$
\end{tabular}
*$\mu_0$ is the value assumed to be $\mu$ under $H_0$. Interpret this as the \# of $SE$
above/below the mean of the null distribution. \\
The $p$ value is the area under the $t$ distribution, with $n - 1$ degrees of freedom. \\

\subsection{Inference on Multiple Means}
Data can be paired or unpaired. 
Paired data is observations that are similar, and we are interested in differences between them. \\
\textbf{For paired data:} \\
Consider a new distribution of the \textbf{differences} in each pair of observations. \\
Hypothesis testing, confidence intervals, etc. are exectly the same as inference on a single mean, just on the difference between means this time.
\textbf{For unpaired data:} \\ 
If the variance of the 2 distributions is similar, use the 2-sample:
\begin{tabular}{@{}c@{}|@{}c@{}}
        \hline
$SE(\bar{X}-\bar{Y})$ & $SE=\sqrt{\frac{\sum(x_i-\bar{x})^2+\sum(y_i-\bar{y})^2}{n_x+n_y-2}}\cdot\sqrt{\frac1{n_x}+\frac1{n_y}}$   \\
Statistic & $t_{obs}=\frac{(\bar{X}-\bar{Y})-(\mu_{X0}-\mu_{Y0})}{SE(\bar{X}-\bar{Y})}$  \\
Degrees of freedom & $\textstyle DF=n_x+n_y-2$   \\
Interval & $(\bar{X}-\bar{Y})\pm t_{crit}*SE$   \\
\end{tabular} \\
Welch when variance different. \\
\begin{tabular}{@{}c@{}|@{}c@{}}
        \hline
$SE(\bar{X}-\bar{Y})$ & $SE=\sqrt{\frac{s_x^2}{n_x}+\frac{s_y^2}{n_y}}$  \\
Statistic & $t_{obs}=\frac{(\bar{X}-\bar{Y})-(\mu_{X0}-\mu_{Y0})}{SE(\bar{X}-\bar{Y})}$  \\
Degrees of freedom & $DF=\frac{(s_x^2/n_x\,+\,s_y^2/n_y)^2}{(s_x^2/n_x)^2/(n_x-1)\,+\,(s_y^2/n_y)^2/(n_y-1)}$  \\
Interval & $(\bar{X}-\bar{Y})\pm t_{crit}*SE$  \\
\end{tabular}
Where $\mu_{X0} - \mu_{Y0}$ is the difference in means assumed under $H_0$. \\
$p$ process is same as before, find area under $t$ distribution according to $H_a$.
\subsection{T Distributions}
Recall the $T$ statistic for inference on means. \\
$T = \frac{\bar{X} - \mu_0}{s / \sqrt{n}}$, notice that we use $s$, a point estimate for $\sigma$.
This introduces randomness, so $T$ is not quite normally distributed, so we use $t$ distribution. \\
The standard deviation is $\frac{d}{d-2}$, $d > 2$ where $d$ is degrees of freedom. If $d \in \mathbb{Z}$, round it down. 
In practice, the $t$ distribution converges to the normal as $d$ increases. Still, it resembles a stretched normal. 

\section{Regression}
\subsection{Font face}
\newcommand{\FontCmd}[3]{\PBS\verb!\#1{!\textit{text}\verb!}!  \> %
                         \verb!{\#2 !\textit{text}\verb!}! \> %
                         \#1{#3}}
\begin{tabular}{@{}l@{}l@{}l@{}}
\textit{Command} & \textit{Declaration} & \textit{Effect} \\
\verb!\textrm{!\textit{text}\verb!}!                    & %
        \verb!{\rmfamily !\textit{text}\verb!}!               & %
        \textrm{Roman family} \\
\verb!\textsf{!\textit{text}\verb!}!                    & %
        \verb!{\sffamily !\textit{text}\verb!}!               & %
        \textsf{Sans serif family} \\
\verb!\texttt{!\textit{text}\verb!}!                    & %
        \verb!{\ttfamily !\textit{text}\verb!}!               & %
        \texttt{Typewriter family} \\
\verb!\textmd{!\textit{text}\verb!}!                    & %
        \verb!{\mdseries !\textit{text}\verb!}!               & %
        \textmd{Medium series} \\
\verb!\textbf{!\textit{text}\verb!}!                    & %
        \verb!{\bfseries !\textit{text}\verb!}!               & %
        \textbf{Bold series} \\
\verb!\textup{!\textit{text}\verb!}!                    & %
        \verb!{\upshape !\textit{text}\verb!}!               & %
        \textup{Upright shape} \\
\verb!\textit{!\textit{text}\verb!}!                    & %
        \verb!{\itshape !\textit{text}\verb!}!               & %
        \textit{Italic shape} \\
\verb!\textsl{!\textit{text}\verb!}!                    & %
        \verb!{\slshape !\textit{text}\verb!}!               & %
        \textsl{Slanted shape} \\
\verb!\textsc{!\textit{text}\verb!}!                    & %
        \verb!{\scshape !\textit{text}\verb!}!               & %
        \textsc{Small Caps shape} \\
\verb!\emph{!\textit{text}\verb!}!                      & %
        \verb!{\em !\textit{text}\verb!}!               & %
        \emph{Emphasized} \\
\verb!\textnormal{!\textit{text}\verb!}!                & %
        \verb!{\normalfont !\textit{text}\verb!}!       & %
        \textnormal{Document font} \\
\verb!\underline{!\textit{text}\verb!}!                 & %
                                                        & %
        \underline{Underline}
\end{tabular}

The command (t\textit{tt}t) form handles spacing better than the
declaration (t{\itshape tt}t) form.

\subsection{Font size}
\setlength{\columnsep}{14pt} % Need to move columns apart a little
\begin{multicols}{2}
\begin{tabbing}
\verb!\footnotesize!          \= \kill
\verb!\tiny!                  \>  \tiny{tiny} \\
\verb!\scriptsize!            \>  \scriptsize{scriptsize} \\
\verb!\footnotesize!          \>  \footnotesize{footnotesize} \\
\verb!\small!                 \>  \small{small} \\
\verb!\normalsize!            \>  \normalsize{normalsize} \\
\verb!\large!                 \>  \large{large} \\
\verb!\Large!                 \=  \Large{Large} \\  % Tab hack for new column
\verb!\LARGE!                 \>  \LARGE{LARGE} \\
\verb!\huge!                  \>  \huge{huge} \\
\verb!\Huge!                  \>  \Huge{Huge}
\end{tabbing}
\end{multicols}
\setlength{\columnsep}{1pt} % Set column separation back

These are declarations and should be used in the form
\verb!{\small! \ldots\verb!}!, or without braces to affect the entire
document.


\subsection{Verbatim text}

\settowidth{\MyLen}{\texttt{.begin.verbatim..} }
\begin{tabular}{@{}p{\the\MyLen}%
                @{}p{\linewidth-\the\MyLen}@{}}
\verb@\begin{verbatim}@ & Verbatim environment. \\
\verb@\begin{verbatim*}@ & Spaces are shown as \verb*@ @. \\
\verb@\verb!text!@ & Text between the delimiting characters (in this case %
                      `\texttt{!}') is verbatim.
\end{tabular}


\subsection{Justification}
\begin{tabular}{@{}ll@{}}
\textit{Environment}  &  \textit{Declaration}  \\
\verb!\begin{center}!      & \verb!\centering!  \\
\verb!\begin{flushleft}!  & \verb!\raggedright! \\
\verb!\begin{flushright}! & \verb!\raggedleft!  \\
\end{tabular}

\subsection{Miscellaneous}
\verb!\linespread{!$x$\verb!}! changes the line spacing by the
multiplier $x$.





\section{Text-mode symbols}

\subsection{Symbols}
\begin{tabular}{@{}l@{\hspace{1em}}l@{\hspace{2em}}l@{\hspace{1em}}l@{\hspace{2em}}l@{\hspace{1em}}l@{\hspace{2em}}l@{\hspace{1em}}l@{}}
\&              &  \verb!\&! &
\_              &  \verb!\_! &
\ldots          &  \verb!\ldots! &
\textbullet     &  \verb!\textbullet! \\
\$              &  \verb!\$! &
\^{}            &  \verb!\^{}! &
\textbar        &  \verb!\textbar! &
\textbackslash  &  \verb!\textbackslash! \\
\%              &  \verb!\%! &
\~{}            &  \verb!\~{}! &
\#              &  \verb!\#! &
\S              &  \verb!\S! \\
\end{tabular}

\subsection{Accents}
\begin{tabular}{@{}l@{\ }l|l@{\ }l|l@{\ }l|l@{\ }l|l@{\ }l@{}}
\`o   & \verb!\`o! &
\'o   & \verb!\'o! &
\^o   & \verb!\^o! &
\~o   & \verb!\~o! &
\=o   & \verb!\=o! \\
\.o   & \verb!\.o! &
\"o   & \verb!\"o! &
\c o  & \verb!\c o! &
\v o  & \verb!\v o! &
\H o  & \verb!\H o! \\
\c c  & \verb!\c c! &
\d o  & \verb!\d o! &
\b o  & \verb!\b o! &
\t oo & \verb!\t oo! &
\oe   & \verb!\oe! \\
\OE   & \verb!\OE! &
\ae   & \verb!\ae! &
\AE   & \verb!\AE! &
\aa   & \verb!\aa! &
\AA   & \verb!\AA! \\
\o    & \verb!\o! &
\O    & \verb!\O! &
\l    & \verb!\l! &
\L    & \verb!\L! &
\i    & \verb!\i! \\
\j    & \verb!\j! &
!`    & \verb!~`! &
?`    & \verb!?`! &
\end{tabular}


\subsection{Delimiters}
\begin{tabular}{@{}l@{\ }ll@{\ }ll@{\ }ll@{\ }ll@{\ }ll@{\ }l@{}}
`       & \verb!`!  &
``      & \verb!``! &
\{      & \verb!\{! &
\lbrack & \verb![! &
(       & \verb!(! &
\textless  &  \verb!\textless! \\
'       & \verb!'!  &
''      & \verb!''! &
\}      & \verb!\}! &
\rbrack & \verb!]! &
)       & \verb!)! &
\textgreater  &  \verb!\textgreater! \\
\end{tabular}

\subsection{Dashes}
\begin{tabular}{@{}llll@{}}
\textit{Name} & \textit{Source} & \textit{Example} & \textit{Usage} \\
hyphen  & \verb!-!   & X-ray          & In words. \\
en-dash & \verb!--!  & 1--5           & Between numbers. \\
em-dash & \verb!---! & Yes---or no?    & Punctuation.
\end{tabular}


\subsection{Line and page breaks}
\settowidth{\MyLen}{\texttt{.pagebreak} }
\begin{tabular}{@{}p{\the\MyLen}%
                @{}p{\linewidth-\the\MyLen}@{}}
\verb!\\!          &  Begin new line without new paragraph.  \\
\verb!\\*!         &  Prohibit pagebreak after linebreak. \\
\verb!\kill!       &  Don't print current line. \\
\verb!\pagebreak!  &  Start new page. \\
\verb!\noindent!   &  Do not indent current line.
\end{tabular}


\subsection{Miscellaneous}
\settowidth{\MyLen}{\texttt{.rule.w..h.} }
\begin{tabular}{@{}p{\the\MyLen}%
                @{}p{\linewidth-\the\MyLen}@{}}
\verb!\today!  &  \today. \\
\verb!$\sim$!  &  Prints $\sim$ instead of \verb!\~{}!, which makes \~{}. \\
\verb!~!       &  Space, disallow linebreak (\verb!W.J.~Clinton!). \\
\verb!\@.!     &  Indicate that the \verb!.! ends a sentence when following
                        an uppercase letter. \\
\verb!\hspace{!$l$\verb!}! & Horizontal space of length $l$
                                (Ex: $l=\mathtt{20pt}$). \\
\verb!\vspace{!$l$\verb!}! & Vertical space of length $l$. \\
\verb!\rule{!$w$\verb!}{!$h$\verb!}! & Line of width $w$ and height $h$. \\
\end{tabular}



\section{Tabular environments}

\subsection{\texttt{tabbing} environment}
\begin{tabular}{@{}l@{\hspace{1.5ex}}l@{\hspace{10ex}}l@{\hspace{1.5ex}}l@{}}
\verb!\=!  &   Set tab stop. &
\verb!\>!  &   Go to tab stop.
\end{tabular}

Tab stops can be set on ``invisible'' lines with \verb!\kill!
at the end of the line.  Normally \verb!\\! is used to separate lines.


\subsection{\texttt{tabular} environment}
\verb!\begin{array}[!\textit{pos}\verb!]{!\textit{cols}\verb!}!   \\
\verb!\begin{tabular}[!\textit{pos}\verb!]{!\textit{cols}\verb!}! \\
\verb!\begin{tabular*}{!\textit{width}\verb!}[!\textit{pos}\verb!]{!\textit{cols}\verb!}!


\subsubsection{\texttt{tabular} column specification}
\settowidth{\MyLen}{\texttt{p}\{\textit{width}\} \ }
\begin{tabular}{@{}p{\the\MyLen}@{}p{\linewidth-\the\MyLen}@{}}
\texttt{l}    &   Left-justified column.  \\
\texttt{c}    &   Centered column.  \\
\texttt{r}    &   Right-justified column. \\
\verb!p{!\textit{width}\verb!}!  &  Same as %
                              \verb!\parbox[t]{!\textit{width}\verb!}!. \\ 
\verb!@{!\textit{decl}\verb!}!   &  Insert \textit{decl} instead of
                                    inter-column space. \\
\verb!|!      &   Inserts a vertical line between columns. 
\end{tabular}


\subsubsection{\texttt{tabular} elements}
\settowidth{\MyLen}{\texttt{.cline.x-y..}}
\begin{tabular}{@{}p{\the\MyLen}@{}p{\linewidth-\the\MyLen}@{}}
\verb!\hline!           &  Horizontal line between rows.  \\
\verb!\cline{!$x$\verb!-!$y$\verb!}!  &
                        Horizontal line across columns $x$ through $y$. \\
\verb!\multicolumn{!\textit{n}\verb!}{!\textit{cols}\verb!}{!\textit{text}\verb!}! \\
        &  A cell that spans \textit{n} columns, with \textit{cols} column specification.
\end{tabular}

\section{Math mode}
For inline math, use \verb!\(...\)! or \verb!$...$!.
For displayed math, use \verb!\[...\]! or \verb!\begin{equation}!.

\begin{tabular}{@{}l@{\hspace{1em}}l@{\hspace{2em}}l@{\hspace{1em}}l@{}}
Superscript$^{x}$       &
\verb!^{x}!             &  
Subscript$_{x}$         &
\verb!_{x}!             \\  
$\frac{x}{y}$           &
\verb!\frac{x}{y}!      &  
$\sum_{k=1}^n$          &
\verb!\sum_{k=1}^n!     \\  
$\sqrt[n]{x}$           &
\verb!\sqrt[n]{x}!      &  
$\prod_{k=1}^n$         &
\verb!\prod_{k=1}^n!    \\ 
\end{tabular}

\subsection{Math-mode symbols}

% The ordering of these symbols is slightly odd.  This is because I had to put all the
% long pieces of text in the same column (the right) for it all to fit properly.
% Otherwise, it wouldn't be possible to fit four columns of symbols here.

\begin{tabular}{@{}l@{\hspace{1ex}}l@{\hspace{1em}}l@{\hspace{1ex}}l@{\hspace{1em}}l@{\hspace{1ex}} l@{\hspace{1em}}l@{\hspace{1ex}}l@{}}
$\leq$          &  \verb!\leq!  &
$\geq$          &  \verb!\geq!  &
$\neq$          &  \verb!\neq!  &
$\approx$       &  \verb!\approx!  \\
$\times$        &  \verb!\times!  &
$\div$          &  \verb!\div!  &
$\pm$           & \verb!\pm!  &
$\cdot$         &  \verb!\cdot!  \\
$^{\circ}$      & \verb!^{\circ}! &
$\circ$         &  \verb!\circ!  &
$\prime$        & \verb!\prime!  &
$\cdots$        &  \verb!\cdots!  \\
$\infty$        & \verb!\infty!  &
$\neg$          & \verb!\neg!  &
$\wedge$        & \verb!\wedge!  &
$\vee$          & \verb!\vee!  \\
$\supset$       & \verb!\supset!  &
$\forall$       & \verb!\forall!  &
$\in$           & \verb!\in!  &
$\rightarrow$   &  \verb!\rightarrow! \\
$\subset$       & \verb!\subset!  &
$\exists$       & \verb!\exists!  &
$\notin$        & \verb!\notin!  &
$\Rightarrow$   &  \verb!\Rightarrow! \\
$\cup$          & \verb!\cup!  &
$\cap$          & \verb!\cap!  &
$\mid$          & \verb!\mid!  &
$\Leftrightarrow$   &  \verb!\Leftrightarrow! \\
$\dot a$        & \verb!\dot a!  &
$\hat a$        & \verb!\hat a!  &
$\bar a$        & \verb!\bar a!  &
$\tilde a$      & \verb!\tilde a!  \\

$\alpha$        &  \verb!\alpha!  &
$\beta$         &  \verb!\beta!  &
$\gamma$        &  \verb!\gamma!  &
$\delta$        &  \verb!\delta!  \\
$\epsilon$      &  \verb!\epsilon!  &
$\zeta$         &  \verb!\zeta!  &
$\eta$          &  \verb!\eta!  &
$\varepsilon$   &  \verb!\varepsilon!  \\
$\theta$        &  \verb!\theta!  &
$\iota$         &  \verb!\iota!  &
$\kappa$        &  \verb!\kappa!  &
$\vartheta$     &  \verb!\vartheta!  \\
$\lambda$       &  \verb!\lambda!  &
$\mu$           &  \verb!\mu!  &
$\nu$           &  \verb!\nu!  &
$\xi$           &  \verb!\xi!  \\
$\pi$           &  \verb!\pi!  &
$\rho$          &  \verb!\rho!  &
$\sigma$        &  \verb!\sigma!  &
$\tau$          &  \verb!\tau!  \\
$\upsilon$      &  \verb!\upsilon!  &
$\phi$          &  \verb!\phi!  &
$\chi$          &  \verb!\chi!  &
$\psi$          &  \verb!\psi!  \\
$\omega$        &  \verb!\omega!  &
$\Gamma$        &  \verb!\Gamma!  &
$\Delta$        &  \verb!\Delta!  &
$\Theta$        &  \verb!\Theta!  \\
$\Lambda$       &  \verb!\Lambda!  &
$\Xi$           &  \verb!\Xi!  &
$\Pi$           &  \verb!\Pi!  &
$\Sigma$        &  \verb!\Sigma!  \\
$\Upsilon$      &  \verb!\Upsilon!  &
$\Phi$          &  \verb!\Phi!  &
$\Psi$          &  \verb!\Psi!  &
$\Omega$        &  \verb!\Omega!  
\end{tabular}
\footnotesize

%\subsection{Special symbols}
%\begin{tabular}{@{}ll@{}}
%$^{\circ}$  &  \verb!^{\circ}! Ex: $22^{\circ}\mathrm{C}$: \verb!$22^{\circ}\mathrm{C}$!.
%\end{tabular}

\section{Bibliography and citations}
When using \BibTeX, you need to run \texttt{latex}, \texttt{bibtex},
and \texttt{latex} twice more to resolve dependencies.

\subsection{Citation types}
\settowidth{\MyLen}{\texttt{.shortciteN.key..}}
\begin{tabular}{@{}p{\the\MyLen}@{}p{\linewidth-\the\MyLen}@{}}
\verb!\cite{!\textit{key}\verb!}!       &
        Full author list and year. (Watson and Crick 1953) \\
\verb!\citeA{!\textit{key}\verb!}!      &
        Full author list. (Watson and Crick) \\
\verb!\citeN{!\textit{key}\verb!}!      &
        Full author list and year. Watson and Crick (1953) \\
\verb!\shortcite{!\textit{key}\verb!}!  &
        Abbreviated author list and year. ? \\
\verb!\shortciteA{!\textit{key}\verb!}! &
        Abbreviated author list. ? \\
\verb!\shortciteN{!\textit{key}\verb!}! &
        Abbreviated author list and year. ? \\
\verb!\citeyear{!\textit{key}\verb!}!   &
        Cite year only. (1953) \\
\end{tabular}

All the above have an \texttt{NP} variant without parentheses;
Ex. \verb!\citeNP!.


\subsection{\BibTeX\ entry types}
\settowidth{\MyLen}{\texttt{.mastersthesis.}}
\begin{tabular}{@{}p{\the\MyLen}@{}p{\linewidth-\the\MyLen}@{}}
\verb!@article!         &  Journal or magazine article. \\
\verb!@book!            &  Book with publisher. \\
\verb!@booklet!         &  Book without publisher. \\
\verb!@conference!      &  Article in conference proceedings. \\
\verb!@inbook!          &  A part of a book and/or range of pages. \\
\verb!@incollection!    &  A part of book with its own title. \\
%\verb!@manual!          &  Technical documentation. \\
%\verb!@mastersthesis!   &  Master's thesis. \\
\verb!@misc!            &  If nothing else fits. \\
\verb!@phdthesis!       &  PhD. thesis. \\
\verb!@proceedings!     &  Proceedings of a conference. \\
\verb!@techreport!      &  Tech report, usually numbered in series. \\
\verb!@unpublished!     &  Unpublished. \\
\end{tabular}

\subsection{\BibTeX\ fields}
\settowidth{\MyLen}{\texttt{organization.}}
\begin{tabular}{@{}p{\the\MyLen}@{}p{\linewidth-\the\MyLen}@{}}
\verb!address!         &  Address of publisher.  Not necessary for major
                                publishers.  \\
\verb!author!           &  Names of authors, of format .... \\
\verb!booktitle!        &  Title of book when part of it is cited. \\
\verb!chapter!          &  Chapter or section number. \\
\verb!edition!          &  Edition of a book. \\
\verb!editor!           &  Names of editors. \\
\verb!institution!      &  Sponsoring institution of tech.\ report. \\
\verb!journal!          &  Journal name. \\
\verb!key!              &  Used for cross ref.\ when no author. \\
\verb!month!            &  Month published. Use 3-letter abbreviation. \\
\verb!note!             &  Any additional information. \\
\verb!number!           &  Number of journal or magazine. \\
\verb!organization!     &  Organization that sponsors a conference. \\
\verb!pages!            &  Page range (\verb!2,6,9--12!). \\
\verb!publisher!        &  Publisher's name. \\
\verb!school!           &  Name of school (for thesis). \\
\verb!series!           &  Name of series of books. \\
\verb!title!            &  Title of work. \\
\verb!type!             &  Type of tech.\ report, ex. ``Research Note''. \\
\verb!volume!           &  Volume of a journal or book. \\
\verb!year!             &  Year of publication. \\
\end{tabular}
Not all fields need to be filled.  See example below.

\subsection{Common \BibTeX\ style files}
\begin{tabular}{@{}l@{\hspace{1em}}l@{\hspace{3em}}l@{\hspace{1em}}l@{}}
\verb!abbrv!    &  Standard &
\verb!abstract! &  \texttt{alpha} with abstract \\
\verb!alpha!    &  Standard &
\verb!apa!      &  APA \\
\verb!plain!    &  Standard &
\verb!unsrt!    &  Unsorted \\
\end{tabular}

The \LaTeX\ document should have the following two lines just before
\verb!\end{document}!, where \verb!bibfile.bib! is the name of the
\BibTeX\ file.
\begin{verbatim}
\bibliographystyle{plain}
\bibliography{bibfile}
\end{verbatim}

\subsection{\BibTeX\ example}
The \BibTeX\ database goes in a file called
\textit{file}\texttt{.bib}, which is processed with \verb!bibtex file!. 
\begin{verbatim}
@String{N = {Na\-ture}}
@Article{WC:1953,
  author  = {James Watson and Francis Crick},
  title   = {A structure for Deoxyribose Nucleic Acid},
  journal = N,
  volume  = {171},
  pages   = {737},
  year    = 1953
}
\end{verbatim}


\section{Sample \LaTeX\ document}
\begin{verbatim}
\documentclass[11pt]{article}
\usepackage{fullpage}
\title{Template}
\author{Name}
\begin{document}
\maketitle

\section{section}
\subsection*{subsection without number}
text \textbf{bold text} text. Some math: $2+2=5$
\subsection{subsection}
text \emph{emphasized text} text. \cite{WC:1953}
discovered the structure of DNA.

A table:
\begin{table}[!th]
\begin{tabular}{|l|c|r|}
\hline
first  &  row  &  data \\
second &  row  &  data \\
\hline
\end{tabular}
\caption{This is the caption}
\label{ex:table}
\end{table}

The table is numbered \ref{ex:table}.
\end{document}
\end{verbatim}

\end{multicols}
\end{document}
